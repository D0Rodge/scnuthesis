%%% Local Variables:
%%% mode: latex
%%% TeX-master: "../main"
%%% End:

\appendixsection{外文资料原文}
\label{cha:engorg}
\appendixitem{first principles}

\textbf{Typography exists to honor content.}

Like oratory, music, dance, calligraphy -- like anything that lends its grace to language -- typography is an art that can be deliberately misused. It is a craft by which the meanings of a text (or its absence of meaning) can be clarified, honored and shared, or knowingly disguised.

In a world rife with unsolicited messages, typography must often draw attention to itself before it will be read. Yet in order to be read, it must relinquish the attention it has drawn. Typography with anything to say therefore aspires to a kind of statuesque transparency. Its other traditional role is durability: not immunity to change, but a clear superiority to fashion. Typography at its best is a visual form of language linking timelessness and time.

One of the principles of durable typography is always legibility; another is something more than legibility: some earned or unearned interest that gives its living energy to the page. It takes various forms and goes by various names, including serenity, liveliness, grace and joy.

These principles apply, in different ways, to the typography of business cards, instruction sheets and postage stamps, as well as to editions of religious scriptures, literary classics and other books that aspire to join their ranks. Within limits, the same principles apply even to stock market reports, airline schedules, milk cartons, classified ads. But laughter, grace and joy, like legibility itself, all feed on meaning, which the writer, the words and the subject, not the typographer, must generally provide.

In 1770, a bill was introduced in the English Parliament with the following provisions:
\begin{quote}$\ldots$ all women of whatever age, rank, profession, or degree, whether virgins, maids, or widows, that shall $\ldots$ impose upon, seduce, and betray into matrimony, any of His Majesty's subjects, by the scents, paints, cosmetic washes, artificial teeth, false hair, Spanish wool, iron stays, hoops, high heeled shoes {\rm [}or{\rm ]} bolstered hips shall incur the pen\-alty of the law in force against witchcraft $\ldots$ and $\ldots$ the marriage, upon conviction, shall stand null and void.
\end{quote}
The function of typography, as I understand it, is neither to further the power of witches nor to bolster the defenses of those, like this unfortunate parliamentarian, who live in terror of being tempted and deceived. The satisfactions of the craft come from elucidating, and perhaps even ennobling, the text, not from deluding the unwary reader by applying scents, paints and iron stays to empty prose. But humble texts, such as classified ads or the telephone directory, may profit as much as anything else from a good typographical bath and a change of clothes. And many a book, like many a warrior or dancer or priest of either sex, may look well with some paint on its face, or indeed with a bone in its nose.

\textbf{Letters have a life and dignity of their own.}

Letterforms that honor and elucidate what humans see and say deserve to be honored in their turn. Well-chosen words deserve well-chosen letters; these in their turn deserve to be set with affection, intelligence, knowledge and skill. Typography is a link, and it ought, as a matter of honor, courtesy and pure delight, to be as strong as the others in the chain.

Writing begins with the making of footprints, the leaving of sighs. Like speaking, it is a perfectly natural act which humans have carried to complex extremes. The typographer's task has always been to add a somewhat unnatural edge, a protective shell of artificial order, to the power of the writing hand. The tools have altered over the centuries, and the exact degree of unnaturalness desired has varied from place to place and time to time, but the character of the essential transformation between manuscript and type has scarcely changed.

The original purpose of type was simply copying. The job of the typographer was to imitate the scribal hand in a form that permitted exact and fast replication. Dozens, then hundreds, then thousands of copies were printed in less time than a scribe would need to finish one. This excuse for setting texts in type has disappeared. In the age of photolithography, digital scanning and offset printing, it is as easy to print directly from handwritten copy as from text that is typographically composed. Yet the typographer's task has little changed. It is still to give the illusion of superhuman speed and stamina -- and of superhuman patience and precision~-- to the writing hand.

Typography is just that: idealized writing. Writers themselves now rarely have the calligraphic skill of earlier scribes, but they evoke countless versions of ideal script by their varying voices and literary styles. To these blind and often invisible visions, the typographer must respond in visible terms.

\appendixitem{second principles}

\textbf{Typography exists to honor content.}

nothing
