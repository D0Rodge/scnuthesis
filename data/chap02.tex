%!TEX root = ../thesis.tex
\chapter{论文正文}
\label{chap:main}
本章将进入论文排版的正文
主要包括:字体段落,图片表格,公式定理,参考文献四部分。
版面将包括在\scnuthesis{}中使用到的所有格式,模板中自定义命令到或者特有的东西都将被一一介绍,
希望大家能看到对自己有用的东西,方便上手。

为了节省时间,这份说明的内容同样参考了国防科技大学的\nudtpaper{}模板~,在此表示感谢。

\section{字体段落}
\label{sec:font}

本节内容来自华南师范大学外部门户\footnote{外部门户主页:\url{http://www.scnu.edu.cn/}}。

华南师范大学始建于1933年,是一所学科门类齐全的国家“211 工程”重点建设大学和广东%
省省属重点大学。学校现有广州石牌、广州大学城和南海3个校区,占地面积共3079亩,%
校舍面积共 126万平方米。校园环境优美,景色怡人~,人文景观遍布,文化气息浓厚,为广%
大师生提供了良好的学习、工作和生活环境。

学校现有4个国家重点学科(含1个重点培育学科), 9个国家“211 工程”重点建设学%
科,2个广东省一级学科重点学科,11个广东省二级学科重点学科。有71个本科专业,有14个%
博士学位授权一级学科、90个博士学位授权点(未含2011年8月一级学科调整后新增的博士%
点)、1个博士专业学位授权点,33个硕士学位授权一级学科、174个硕士学位授权点(未%
含2011年8月一级学科调整后新增的硕士点)、10个硕士专业学位授权点,涉及到哲学、经济%
学、法学、教育学、文学、历史学、理学、工学、管理学、农学、医学等11个学科门%
类。 有12个博士后流动站。

学校拥有一批实力较强的实验室和科研基地。有教育部激光生命科学重点实验室、环境理论
化学重点实验室(省部共建)、教育部电化学储能材料与技术工程研究中心、卫生部(中医
药管理局)中医药与光子技术实验室、国家理科基础科学研究和教学人才培养基地、教育部
部省共建人文社科重点研究基地(心理应用研究中心)、国家体育总局重点研究基地(体育
社会科学研究基地),有7个广东省重点实验室(中心),6个广东省高校科研型重点实验
室,3个广东省高校产学研结合示范基地,6个广东省普通高校人文社会科学重点研究基
地,1个广东省高校工程技术研究中心。学校还拥有“物理学科基础课”、“信息传
播”~、“心理学”等3个国家级实验教学示范中心,7个广东省实验教学示范中心。此外,教
育部高校辅导员培训和研修基地、广东省普通高等学校师资培训中心~、广东省网络图书馆、
广东高校建筑规划设计院等机构均设在学校。

70 多年来,学校数易校名,几度迁徙,虽历经沧桑,却弦歌不辍。一代又一代华师人秉承勷
勤大学师范学院{\kai “研究高深学术,养成社会之专门人才”}的优良传统,承传南方大
学{\kai “忠诚团结,朴实虚心,勤劳勇敢,实事求是”}的革命精神~,践行{\kai “艰苦奋斗、严谨治
学、求实创新、为人师表”}的校训,筚路蓝缕,薪火相传,共同铸就了学校今天的繁荣与
发展。

华南师范大学历史上的名字包括:

\begin{itemize}
\item[楷体] {\kai 广州市立师范学校}
\item[黑体] {\hei 勷勤大学师范学院}
\item[隶书] {\li 勷勤大学教育学院}
\item[宋体] {\song 广东省立教育学院}
\item[仿宋] {\fs 广东省立文理学院}
\item[粗体] {\bfseries 广东省文理学院}
\item[斜体] {\itshape 华南师范学院}
\item[粗斜体] {\bfseries\itshape 广东师范学院}
\item[字体颜色] {\textcolor{red}{华南师范大学}}
\end{itemize}

上面这段内容使用了itemize列表环境,\LaTeX{}默认的列表环境会在条目之间插入
过多的行距,若用户需要紧凑的行距,可以使用compactitem环境。

下面测试英文字体:

Remember the \textsf{more} \textbf{font} {\bfseries\tiny you \sffamily use,
\Large the \scshape more \itshape beautiful \slshape your \footnotesize
document becomes.}

\begin{itemize}
\item[英文黑体] Typeset text in \textbf{bold} series
\item[英文斜体] Typeset text in \textit{italic} shape
\item[Roman字体] Typeset text in roman family
\item[Sans Serif字体] Typeset text in \textsf{sans serif} family
\item[typewriter字体] Typeset text in \texttt{typewriter} family
\end{itemize}

下面测试字号:
\begin{itemize}
\item[初号] {\chuhao 华南师范大学}
\item[小初] {\xiaochu 华南师范大学}
\item[一号] {\yihao 华南师范大学}
\item[小一] {\xiaoyi 华南师范大学}
\item[二号] {\erhao 华南师范大学}
\item[小二] {\xiaoer 华南师范大学}
\item[三号] {\sanhao 华南师范大学}
\item[小三] {\xiaosan 华南师范大学}
\item[四号] {\sihao 华南师范大学}
\item[小四] {\xiaosi 华南师范大学}
\item[五号] {\wuhao 华南师范大学}
\item[小五] {\xiaowu 华南师范大学}
\end{itemize}

\section{表格明细}
\label{sec:figure}
表格是书中的重要组成部分,这一章将从简单的表格讲起,到复杂的表格为止。

\subsection{基本表格}
\label{sec:basictable}

模板中关于表格的宏包有三个: \textsf{booktabs}、\textsf{array} 和
\textsf{longtabular},命令有一个 \verb|\hlinewd|。三线表建议使用\textsf{booktabs}中提供的,
包含toprule、midrule 和 bottomrule三条命令。
它们与\textsf{longtable} 能很好的配合使用。如果表格比较简单的话可以直接用命令
\verb|hlinewd{xpt}| 控制。下面来看一个表格:
\begin{table}[htb]
  \centering
  \begin{minipage}[t]{0.8\linewidth} % 如果想在表格中使用脚注,minipage是个不错的办法
  \caption[模板文件]{模板文件。如果表格的标题很长,那么在表格索引中就会很不美
    观,所以要像 chapter 那样在前面用中括号写一个简短的标题。这个标题会出现在索
    引中。}
  \label{tab:template-files}
    \begin{tabular*}{\linewidth}{lp{10cm}}
      \toprule[1.5pt]
      {\hei 文件名} & {\hei 描述} \\
      \midrule[1pt]
      scnuthesis.ins & \LaTeX{} 安装文件,docstrip\footnote{表格中的脚注} \\
      scnuthesis.dtx & 所有的一切都在这里面\footnote{再来一个}。\\
      scnuthesis.cls & 模板类文件。\\
      scnuthesis.cfg & 模板配置文。cls 和 cfg 由前两个文件生成。\\
      bstutf8.bst   & 参考文献 Bibtex 样式文件。\\
      myscnu.sty    & 常用的包和命令写在这里,减轻主文件的负担。\\
      \bottomrule[1.5pt]
    \end{tabular*}
  \end{minipage}
\end{table}

如果你不需要在表格中插入脚注,可以将minipage环境去掉。

表 \ref{tab:template-files} 列举了本模板主要文件及其功能。
请大家注意三线表中各条线对应的命令。这个例子还展示了如何在表格中正确使用脚注。
由于 \LaTeX{} 本身不支持在表格中使用 \verb|\footnote|,所以我们不得不将表格放在
小页中,而且最好将表格的宽度设置为小页的宽度,这样脚注看起来才更美观。

如果遇到表格内容自动调整的问题,可以有两种解决办法: 其一就是用\verb|tabular*|,在
两列之间全部插入空白,该方法存在的缺点是当只有两列需要自动调整时,若小页预留空间
过大,可能插入过多的\verb|fill|,导致不美观; 另外一种就是使用\verb|tabularx|自动
调整了,需要定制一个\textbf{Z}环境,在新版本中,该命令已添加到\verb|myscnu.sty|中。
下面是这两个方法实现的对比,各位可以仔细对比一下,推荐使用后者。

\begin{table}[htbp]
\centering
\begin{minipage}[t]{0.9\linewidth}
\caption{Reed Solomon码的典型应用}
\label{tab:RSused}
\begin{tabular*}{\linewidth}{c @{\extracolsep{\fill}} c}
\toprule[1.5pt]
{\hei 应用领域} & {\hei 编码方案}\\
\midrule[1pt]
磁盘驱动器 & RS(32,28,5)码 \footnote{码长为32、维数为28、最小距离为5} \\
CD & 交叉交织RS码(CIRC) \\
DVD & RS(208,192,17)码、RS(182,172,11)码 \\
光纤通信 & RS(255,229,17)码 \\
\bottomrule[1.5pt]
\end{tabular*}
\end{minipage}
\end{table}

\begin{table}[htbp]
\centering
\begin{minipage}[t]{0.9\linewidth}
\caption{Reed Solomon码的典型应用}
\label{tab:RSuse}
\begin{tabularx}{\linewidth}{cZ}
\toprule[1.5pt]
{\hei 应用领域} & {\hei 编码方案}\\
\midrule[1pt]
磁盘驱动器 & RS(32,28,5)码 \footnote{码长为32、维数为28、最小距离为5} \\
CD & 交叉交织RS码(CIRC) \\
DVD & RS(208,192,17)码、RS(182,172,11)码 \\
光纤通信 & RS(255,229,17)码 \\
\bottomrule[1.5pt]
\end{tabularx}
\end{minipage}
\end{table}

\subsection{复杂表格}
\label{sec:complicatedtable}

我们经常会在表格下方标注数据来源,或者对表格里面的条目进行解释。前面的脚注是一种
不错的方法,如果你不喜欢脚注。那么完全可以在表格后面自己写注释,比如表~\ref{tab:tabexamp1}。
\begin{table}[htbp]
  \centering
  \caption{复杂表格示例 1}
  \label{tab:tabexamp1}
  \begin{minipage}[t]{0.8\textwidth} 
    \begin{tabularx}{\linewidth}{|l|X|X|X|X|}
      \hline
      \multirow{2}*{\backslashbox{x}{y}}  & \multicolumn{2}{c|}{First Half} & \multicolumn{2}{c|}{Second Half}\\
      \cline{2-5}
      & 1st Qtr &2nd Qtr&3rd Qtr&4th Qtr \\ 
      \hline
      East$^{*}$ &   20.4&   27.4&   90&     20.4 \\
      West$^{**}$ &   30.6 &   38.6 &   34.6 &  31.6 \\ 
      \hline
    \end{tabularx}\\[2pt]
    \footnotesize
    *:东部\\
    **:西部
  \end{minipage}
\end{table}

此外,表~\ref{tab:tabexamp1} 同时还演示了另外两个功能:1)通过 \textsf{tabularx} 的
 \texttt{|X|} 扩展实现表格自动放大;2)通过命令 \verb|\backslashbox| 在表头部分
插入反斜线。

为了使我们的例子更接近实际情况,我会在必要的时候插入一些“无关”文字,以免太多图
表同时出现,导致排版效果不太理想。

学校七十余载薪火相传,名师荟萃,著名的教育家罗浚、汪德亮,五四新诗开创者之一康白
情,古代文学家李镜池,古汉语学家吴三立,历史学家王越,逻辑学家李匡武,心理学家阮
镜清,教育学家叶佩华、朱勃,数学家叶述武,物理学家黄友谋、刘颂豪,著名体育教育家
袁浚等众多名家、名师先后在此执教。该校虽数度易名、几经迁徙,但一代又一代华师人秉
承勷勤大学师范学院``研究高深学术,养成社会之专门人才''的优良传统,承传南方大
学``忠诚团结,实事求是''的革命精神,践行``艰苦奋斗、严谨治学、求实创新、为人师
表''的校训,不断推动学校事业向前发展。特别是改革开放以来,抓住科教兴国、人才强国
的发展机遇,凭借建设文化省、教育强省和国家``211工程''的强劲东风,形成了学校现在
跨越式发展的大好局面。

不可否认 \LaTeX{} 的表格功能没有想象中的那么强大,不过只要你足够认真,足够细致,那么
同样可以排出来非常复杂非常漂亮的表格。请参看表~\ref{tab:tabexamp2}。
\begin{table}[htbp]
  \centering\dawu[1.3]
  \caption{复杂表格示例 2}
  \label{tab:tabexamp2}
  \begin{tabular}[c]{|c|m{0.8in}|c|c|c|c|c|}\hline
    \multicolumn{2}{|c|}{Network Topology} & \# of nodes & 
    \multicolumn{3}{c|}{\# of clients} & Server \\\hline
    GT-ITM & Waxman Transit-Stub & 600 &
    \multirow{2}{2em}{2\%}& 
    \multirow{2}{2em}{10\%}& 
    \multirow{2}{2em}{50\%}& 
    \multirow{2}{1.2in}{Max. Connectivity}\\\cline{1-3}
    \multicolumn{2}{|c|}{Inet-2.1} & 6000 & & & &\\\hline
    \multirow{2}{1in}{Xue} & Rui  & Ni &\multicolumn{4}{c|}{\multirow{2}*{\scnuthesis}}\\\cline{2-3}
    & \multicolumn{2}{c|}{ABCDEF} &\multicolumn{4}{c|}{} \\\hline
\end{tabular}
\end{table}

\subsection{子表格与跨页表格}

浮动体的并排放置一般有两种情况:1)二者没有关系,为两个独立的浮动体;2)二者隶属
于同一个浮动体。对表格来说并排表格既可以像图~\ref{tab:parallel1}、图~\ref{tab:parallel2} 
使用小页环境,也可以如图~\ref{tab:subtable} 使用子表格来做。后面我们将讲解图的例子。
\begin{table}[htb]
\noindent\begin{minipage}{0.45\textwidth}
\centering
\caption{第一个并排子表格}
\label{tab:parallel1}
\begin{tabular}{p{2cm}p{2cm}}
\toprule[1.5pt]
111 & 222 \\\midrule[1pt]
222 & 333 \\\bottomrule[1.5pt]
\end{tabular}
\end{minipage}
\begin{minipage}{0.45\textwidth}
\centering
\caption{第二个并排子表格}
\label{tab:parallel2}
\begin{tabular}{p{2cm}p{2cm}}
\toprule[1.5pt]
111 & 222 \\\midrule[1pt]
222 & 333 \\\bottomrule[1.5pt]
\end{tabular}
\end{minipage}
\end{table}

学校教师队伍结构良好、水平较高,拥有一批在国内外具有一定影响的专家学者。现有教师
队伍 1900 多人,其中教授 400 多人,副教授 500 多人,博士、硕士研究生导师 800 多人,
具有博士、硕士学位和研究生学历的1500 多人。在师资队伍中,有中国科学院院士 7 人、
瑞典皇家科学院院士2人、“千人计划”入围者 1人、长江学者4人、获得国家杰出青年基金
项目资助者3人、“新世纪百千万人才”国家级人选 5人、国家级教学名师2 人、广东省领军
人才4人、珠江学者4人、广东省高等学校“千百十工程”国家级培养对象5人,拥有教育
部“长江学者与创新团队发展计划”创新团队 1个、广东省创新科研团队1个,并有国务院学
位委员会学科评议组成员 3 人、教育部高等学校教学指导委员会成员 12 人。

\begin{table}[htbp]
\centering
\caption{并排子表格}
\label{tab:subtable}
\subfloat[第一个子表格]{
\begin{tabular}{p{2cm}p{2cm}}
\toprule[1.5pt]
111 & 222 \\\midrule[1pt]
222 & 333 \\\bottomrule[1.5pt]
\end{tabular}}\hskip2cm
\subfloat[第二个子表格]{
\begin{tabular}{p{2cm}p{2cm}}
\toprule[1.5pt]
111 & 222 \\\midrule[1pt]
222 & 333 \\\bottomrule[1.5pt]
\end{tabular}}
\end{table}

如果您要排版的表格长度超过一页,那么推荐使用 \textsf{longtable} 或者 \textsf{supertabular} 
宏包,表~\ref{tab:performance} 就是 \textsf{longtable} 的简单示例。
\begin{longtable}[c]{c*{6}{r}}
\caption{实验数据}\label{tab:performance}\\
\toprule[1.5pt]
 测试程序 & \multicolumn{1}{c}{正常运行} & \multicolumn{1}{c}{同步}
& \multicolumn{1}{c}{检查点}   & \multicolumn{1}{c}{卷回恢复}
& \multicolumn{1}{c}{进程迁移} & \multicolumn{1}{c}{检查点} 	\\
& \multicolumn{1}{c}{时间 (s)} & \multicolumn{1}{c}{时间 (s)}
& \multicolumn{1}{c}{时间 (s)} & \multicolumn{1}{c}{时间 (s)}
& \multicolumn{1}{c}{时间 (s)} &  文件(KB)			\\
\midrule[1pt]%
\endfirsthead%

\multicolumn{7}{c}{续表~\thetable\hskip1em 实验数据}\\

\toprule[1.5pt]
 测试程序 & \multicolumn{1}{c}{正常运行} & \multicolumn{1}{c}{同步} 
& \multicolumn{1}{c}{检查点}   & \multicolumn{1}{c}{卷回恢复}
& \multicolumn{1}{c}{进程迁移} & \multicolumn{1}{c}{检查点} 	\\
& \multicolumn{1}{c}{时间 (s)} & \multicolumn{1}{c}{时间 (s)}
& \multicolumn{1}{c}{时间 (s)} & \multicolumn{1}{c}{时间 (s)}
& \multicolumn{1}{c}{时间 (s)} &  文件(KB)			\\
\midrule[1pt]%
\endhead%
\hline%

\multicolumn{7}{r}{续下页}%

\endfoot%
\endlastfoot%
CG.A.2 & 23.05   & 0.002 & 0.116 & 0.035 & 0.589 & 32491  \\
CG.A.4 & 15.06   & 0.003 & 0.067 & 0.021 & 0.351 & 18211  \\
CG.A.8 & 13.38   & 0.004 & 0.072 & 0.023 & 0.210 & 9890   \\
CG.B.2 & 867.45  & 0.002 & 0.864 & 0.232 & 3.256 & 228562 \\
CG.B.4 & 501.61  & 0.003 & 0.438 & 0.136 & 2.075 & 123862 \\
CG.B.8 & 384.65  & 0.004 & 0.457 & 0.108 & 1.235 & 63777  \\
MG.A.2 & 112.27  & 0.002 & 0.846 & 0.237 & 3.930 & 236473 \\
MG.A.4 & 59.84   & 0.003 & 0.442 & 0.128 & 2.070 & 123875 \\
MG.A.8 & 31.38   & 0.003 & 0.476 & 0.114 & 1.041 & 60627  \\
MG.B.2 & 526.28  & 0.002 & 0.821 & 0.238 & 4.176 & 236635 \\
MG.B.4 & 280.11  & 0.003 & 0.432 & 0.130 & 1.706 & 123793 \\
MG.B.8 & 148.29  & 0.003 & 0.442 & 0.116 & 0.893 & 60600  \\
LU.A.2 & 2116.54 & 0.002 & 0.110 & 0.030 & 0.532 & 28754  \\
LU.A.4 & 1102.50 & 0.002 & 0.069 & 0.017 & 0.255 & 14915  \\
LU.A.8 & 574.47  & 0.003 & 0.067 & 0.016 & 0.192 & 8655   \\
LU.B.2 & 9712.87 & 0.002 & 0.357 & 0.104 & 1.734 & 101975 \\
LU.B.4 & 4757.80 & 0.003 & 0.190 & 0.056 & 0.808 & 53522  \\
LU.B.8 & 2444.05 & 0.004 & 0.222 & 0.057 & 0.548 & 30134  \\
EP.A.2 & 123.81  & 0.002 & 0.010 & 0.003 & 0.074 & 1834   \\
EP.A.4 & 61.92   & 0.003 & 0.011 & 0.004 & 0.073 & 1743   \\
EP.A.8 & 31.06   & 0.004 & 0.017 & 0.005 & 0.073 & 1661   \\
EP.B.2 & 495.49  & 0.001 & 0.009 & 0.003 & 0.196 & 2011   \\
EP.B.4 & 247.69  & 0.002 & 0.012 & 0.004 & 0.122 & 1663   \\
EP.B.8 & 126.74  & 0.003 & 0.017 & 0.005 & 0.083 & 1656   \\
\bottomrule[1.5pt]
\end{longtable}

为了排版方便,这里要插入一些随机的文字,那就加上猩猩博客的东西吧: 
``越来越喜欢吃,自己做的川菜,每次做菜,都像是创作的过程,%
随心所欲; 发现家常菜真的很难做好,越是简单的菜,越是难以做好。
献上一个鱼香肉丝,让我跟随简单的脚步,creat出简约的菜品。''

\subsection{其它}
\label{sec:tableother}
有的同学不想让某个表格或者图片出现在索引里面,那么请使用命令 \verb|\caption*{}|,
这个命令不会给表格编号,也就是出来的只有标题文字而没有“表~XX”,“图~XX”,否则
索引里面序号不连续就显得不伦不类,这也是 \LaTeX{} 里星号命令默认的规则。

\section{绘图插图}

绘图工具分为 GUI 的和 CLI 两种。GUI即是所见即所得的绘图工具~,常见的包
括 Visio、Inkscape、CorelDraw、XFig(jFig)、WinFig、Tpx、Ipe、Dia等;CLI则是需要编
译后才能够得到图形的工具,比较流行的有 PGF/TikZ~、Asymptote、pstricks等。GUI 类绘
图工具比较易于上手,而 CLI 类绘图工具则能够画出更加精确的图形。关于各类绘图工具的
比较和使用方法~,推荐用户到C\TeX{}论坛{\url{http://bbs.ctex.org/}}以及China\TeX{}论坛
{\url{http://bbs.chinatex.org/forum.php}}上的相关板块进行更加深入的了解。

\subsection{插图}
\label{sec:graphs}

强烈推荐《\LaTeXe 插图指南》!关于子图形使用细节请参看\textsf{subfig}手册。 

\subsubsection{一个图形}
\label{sec:onefig}
一般图形都是处在浮动环境中。之所以称为浮动是指最终排版效果图形的位置不一定与源文
件中的位置对应,这也是刚使
用 \LaTeX{} 同学可能遇到的问题。如果要强制固定浮动图形的位置,请使用 \textsf{float} 宏包,
它提供了 \texttt{[H]} 参数,但是除非特别需要,不建议使用\texttt{[H]},
而是倾向于使用\texttt{[htbp]},给\LaTeX{}更多选择。比如图~\ref{fig:ipe}。
\begin{figure}[htbp] % use float package if you want it here
  \centering
  \includegraphics[width=\textwidth]{tikz}
  \caption{利用TikZ制图}
  \label{fig:ipe}
\end{figure}

大学之道,在明明德,在亲民,在止于至善。知止而后有定;定而后能静;静而后能安;安
而后能虑;虑而后能得。物有本末,事有终始。知所先后,则近道矣。古之欲明明德于天
下者,先治其国;欲治其国者,先齐其家;欲齐其家者~,先修其身;欲修其身者,先正其心;
欲正其心者,先诚其意;欲诚其意者~,先致其知;致知在格物。物格而后知至;知至而后
意诚;意诚而后心正;心正而后身修;身修而后家齐;家齐而后国治;国治而后天下
平。自天子以至于庶人,壹是皆以修身为本。其本乱而未治者 否矣。其所厚者薄,而其所
薄者厚,未之有也!

\hfill \pozhehao《大学》

\subsubsection{多个图形}
\label{sec:multifig}

如果多个图形相互独立,并不共用一个图形计数器,那么用 \verb|minipage| 或者
\verb|parbox| 就可以。否则,请参看图~\ref{fig:big1},它包含两个小图,分别是图~\ref{fig:subfig1} 
和图~\ref{fig:subfig2}。推荐使用 \verb|\subfloat|,不要再用
\verb|\subfigure| 和 \verb|\subtable|。
\begin{figure}[htb]
  \centering%
  \subfloat[第一个小图形]{%
    \label{fig:subfig1}
    \includegraphics[height=2cm]{logo.jpg}}\hspace{4em}%
  \subfloat[第二个小图形。如果标题很长的话,它会自动换行,这个 caption 就是这样的例子]{%
    \label{fig:subfig2}
    \includegraphics[height=2cm]{don-hires}}
  \caption{包含子图形的大图形}
  \label{fig:big1}
\end{figure}

培育英才万万千,建设祖国锦绣河山,华师儿女奋勇当先,珠江滚滚红绵艳~,岭南大地草木
春,改革开放阳光好,华师园里花烂漫。

艰苦奋斗众志坚,严谨治学成风范,求实创新勇开拓,为人师表代代相传,教育改革宏图展,
师范园地好摇篮,培育祖国栋梁材,神圣职责我承担。


下面这个例子显示并排$3\times2$的图片,见图\ref{fig:subfig:3x2}:
\begin{figure}[htb]
\centering
\subfloat[]{\includegraphics[width=.27\textwidth]{typography}} \qquad
\subfloat[]{\includegraphics[width=.27\textwidth]{typography}} \qquad
\subfloat[]{\includegraphics[width=.27\textwidth]{typography}} \qquad
\subfloat[]{\includegraphics[width=.27\textwidth]{typography}} \qquad
\subfloat[]{\includegraphics[width=.27\textwidth]{typography}} \qquad
\subfloat[]{\includegraphics[width=.27\textwidth]{typography}}
\caption{并排图片}
\label{fig:subfig:3x2}
\end{figure}

要注意,\texttt{qquad}相当于\verb|\hspace{2em}|,也就是2个字符的宽度,约0.08倍页宽,
图片宽度设定为0.27倍页宽是合适的;在该环境中,尽量不要手动换行。

向前向前向前向前,华师儿女永远向前!

如果要把编号的两个图形并排,那么小页就非常有用了:
\begin{figure}[htb]
\begin{minipage}{0.48\textwidth}
  \centering
  \includegraphics[height=4cm]{building.jpg}
  \caption{并排第一个图}
  \label{fig:parallel1}
\end{minipage}\hfill
\begin{minipage}{0.48\textwidth}
  \centering
  \includegraphics[height=4cm]{cat.jpg}
  \caption{并排第二个图}
  \label{fig:parallel2}
\end{minipage}
\end{figure}

\section{公式定理}
\label{sec:equation}
贝叶斯公式如式~(\ref{equ:chap1:bayes}),其中 $p(y|\mathbf{x})$ 为后验;
$p(\mathbf{x})$ 为先验;分母 $p(\mathbf{x})$ 为归一化因子。
\begin{equation}
\label{equ:chap1:bayes}
p(y|\mathbf{x}) = \frac{p(\mathbf{x},y)}{p(\mathbf{x})}=
\frac{p(\mathbf{x}|y)p(y)}{p(\mathbf{x})} 
\end{equation}

论文里面公式越多,\TeX{} 就越 happy。再看一个 \textsf{amsmath} 的例子:
\newcommand{\envert}[1]{\left\lvert#1\right\rvert} 
\begin{equation}\label{detK2}
\det\mathbf{K}(t=1,t_1,\dots,t_n)=\sum_{I\in\mathbf{n}}(-1)^{\envert{I}}
\prod_{i\in I}t_i\prod_{j\in I}(D_j+\lambda_jt_j)\det\mathbf{A}
^{(\lambda)}(\overline{I}|\overline{I})=0.
\end{equation} 

大家在写公式的时候一定要好好看 \textsf{amsmath} 的文档,并参考模板中的用法:
\begin{multline*}%\tag{[b]} % 这个出现在索引中的
\int_a^b\biggl\{\int_a^b[f(x)^2g(y)^2+f(y)^2g(x)^2]
 -2f(x)g(x)f(y)g(y)\,dx\biggr\}\,dy \\
 =\int_a^b\biggl\{g(y)^2\int_a^bf^2+f(y)^2
  \int_a^b g^2-2f(y)g(y)\int_a^b fg\biggr\}\,dy
\end{multline*}

多列公式也是比较常见的情况,比较常用的办法是用align环境实现:

\begin{equation} 
\mathbf{X} = \left(\begin{array}{ccc} 
x_{11} & x_{12} & \ldots \\ 
x_{21} & x_{22} & \ldots \\ 
\vdots & \vdots & \ddots \end{array} \right) 
\end{equation} 

\begin{equation} 
y = \left\{ \begin{array}{ll} 
a & \textrm{if $d>c$}\\ 
b+x & \textrm{in the morning}\\ 
l & \textrm{all day long} 
\end{array} \right. 
\end{equation} 

\begin{equation} 
\left(\begin{array}{c|c} 
1 & 2 \\ 
\hline 3 & 4 \end{array}\right) 
\end{equation}   

\begin{eqnarray}
f(x) & = & \cos x \\ 
f'(x) & = & -\sin x \\ 
\int_{0}^{x} f(y)\,dy & = & \sin x 
\end{eqnarray} 

{\setlength\arraycolsep{2pt} 
\begin{eqnarray} 
\sin x & = & x -\frac{x^{3}}{3!} +\frac{x^{5}}{5!}-{} \nonumber\\ 
	& & {}-\frac{x^{7}}{7!}+{}\cdots 
\end{eqnarray}} 

另外,\texttt{split}环境可能在XeCJK上不能使用,我们测试一下,看\ref{equ:split}:
\begin{equation}\label{equ:split}
\begin{split}
[z^n]C(z) &= [z^n] \biggl[\frac{e^{3/4}}{\sqrt{1-z}} +
e^{-3/4}(1-z)^{1/2} + \frac{e^{-3/4}}{4}(1-z)^{3/2}
+ O\Bigl( (1-z)^{5/2}\Bigr)\biggr] \\
&= \frac{e^{-3/4}}{\sqrt{\pi n}} - \frac{5e^{-3/4}}{8\sqrt{\pi
n^3}} + \frac{e^{-3/4}}{128 \sqrt{\pi n^5}} +
O\biggl(\frac{1}{\sqrt{\pi
n^7}}\biggr)
\end{split}
\end{equation}
\textbf{注意:} 论文模板中为了与xeCJK稳定版本兼容,调校了split命令,代价是不能在inline使用split命令。

\begin{theorem}
  \label{chapTSthm:rayleigh solution}
  假定 $X$ 的二阶矩存在:
  \begin{equation}
         O_R(\textbf{x},F)=\sqrt{\frac{\textbf{u}_1^T\textbf{A}\textbf{u}_1} {\textbf{u}_1^T\textbf{B}\textbf{u}_1}}=\sqrt{\lambda_1},
  \end{equation}
  其中 $\textbf{A}$ 等于 $(\textbf{x}-EX)(\textbf{x}-EX)^T$,\textbf{B} 表示协方差阵 $E(X-EX)(X-EX)^T$,$\lambda_1$
$\textbf{u}_1$ 是 $\lambda_1$对应的特征向量, $\omega,\ve{\omega},\omegaup,\ve{\omegaup}$.
\end{theorem}

\begin{proof}
 上述优化问题显然是一个 Rayleigh 商问题。我们有
  \begin{align}
     O_R(\textbf{x},F)=\sqrt{\frac{\textbf{u}_1^T\textbf{A}\textbf{u}_1} {\textbf{u}_1^T\textbf{B}\textbf{u}_1}}=\sqrt{\lambda_1},
 \end{align}
 其中 $\lambda_1$ 下列广义特征值问题的最大特征值:
$$
\textbf{A}\textbf{z}=\lambda\textbf{B}\textbf{z}, \textbf{z}\neq 0.
$$
 $\textbf{u}_1$ 是 $\lambda_1$对应的特征向量。结论成立。
\end{proof}

\subsection{非回路故障的推理算法}
我们知道,故障诊断的最终目的,是将故障定位到部件,而由于信号-部件依赖矩阵的存在,因此,实质性的工作是找出由故障部件发出异常信号,
不妨称为源异常信号,而如前所述,源异常信号与异常信号依赖矩阵$\mathbf{S_a}$的全零列是存在一一对应的关系的。因此,我们只要获得了$\mathbf{S_a}$的全零列的相关信息,
也就获得了源异常信号的信息,从而能进一步找到故障源。
通过以上分析,我们构造算法\ref{alg53},用于实现非回路故障诊断。

算法\ref{alg53}中,称$\beta$为源异常信号向量,该向量中与源异常信号对应的元素值为1,其它为0;
称$\gamma$为部件状态向量,该向量中非0元素对应的部件为故障部件~,0元素对应的部件为正常部件。
值得一提的是$\beta$和$\beta_a$的区别。$\beta$指出了源异常信号在所有信号中排序的位置,因此其维数与信号总数相同;
而$\beta_a$指出了源异常信号在所有异常信号中排序的位置,因此其维数与异常信号总数相同。如前所述,信号的“排序”是固定的,
这保证了算法在执行中不出现混乱。
\begin{algorithm}[htbp]
  \caption{非回路故障诊断算法}
  \label{alg53}
  \begin{algorithmic}[1]
    \REQUIRE 信号--部件依赖矩阵$\mathbf{A}$,信号依赖矩阵$\mathbf{S}$,信号状态向量$\alpha$
    \ENSURE 部件状态向量$\gamma$
    \STATE $\mathbf{P}\leftarrow\left(<\alpha>\right)$
    \STATE $\mathbf{S_{a}}\leftarrow\mathbf{P^T}\mathbf{S}\mathbf{P}$
    \FOR{$i=1$ to $S_a$的阶数$m$}
    \STATE $s_i\leftarrow s_i$的第$i$个行向量
    \ENDFOR
    \STATE $\beta_a\leftarrow\lnot \left(s_1\lor s_2\lor \cdots\lor s_m\right)^T$
    \STATE $\beta\leftarrow\mathbf{P}\beta_a$
    \STATE $\gamma\leftarrow\mathbf{A}\beta$
  \end{algorithmic}
\end{algorithm}
\subsubsection{第一类故障回路的推理算法}
第一类故障回路推理与非回路故障推理是算法基本相同,稍微不同的是$\beta_a$的计算。因为第一类故障回路中的信号全部可能是源异常信号,因此我们不必计算
$\beta_a=\lnot \left(\left[s_1\lor s_2\lor \cdots\lor s_m\right]^T\right)$,而直接取$\beta_a=\underbrace{\left[\begin{array}{cccc}1&1&\cdots&1\end{array}\right]^T}_m$,将$\beta_a$代入
算法\ref{alg53},有
\[\beta=\mathbf{P}\beta_a=\mathbf{P}\underbrace{\left[\begin{array}{cccc}1&1&\cdots&1\end{array}\right]^T}_m=\alpha\]
因此一类故障回路的推理算法变得相当简单,例如算法\ref{alg54}
\begin{algorithm}[htbp]
  \caption{第一类故障回路诊断算法}
  \label{alg54}
  \begin{algorithmic}[1]
    \REQUIRE 信号--部件依赖矩阵$\mathbf{A}$,信号状态向量$\alpha$
    \ENSURE 部件状态向量$\gamma$
    \STATE $\gamma\leftarrow\mathbf{A}\alpha$
  \end{algorithmic}
\end{algorithm}

\section{参考文献}
\label{sec:bib}
当然参考文献可以直接写 bibitem,虽然费点功夫,但是好控制,各种格式可以自己随意改
写,在\scnuthesis{}里面,建议使用JabRef编辑和管理文献,再结合\verb|bstutf8.bst|之后,
对中文的支持也很好。

本模板推荐使用 BIB\TeX,样式文件为 bstutf8.bst,基本符合学校的参考文献格式(如专利
等引用未加详细测试)。看看这个例子,关于书的\upcite{tex, companion},
还有这些\upcite{clzs},关于杂志的\upcite{ELIDRISSI94,
  MELLINGER96, SHELL02},硕士论文\upcite{zhubajie, metamori2004},博士论文
\upcite{shaheshang, FistSystem01},标准文件\upcite{IEEE-1363}~,会议论文\upcite{DPMG},%
技术报告\upcite{NPB2}。中文参考文献\upcite{cnarticle}\textsf{特别注意},需要在\verb|bibitem|中
增加\verb|language|域并设为\verb|zh|,英文此项可不填,之后由\verb|bstutf8|统一处理
(具体就是决定一些文献的显示格式,如等、etc)。
若使用\verb|JabRef|,则选择\textsf{Options}$\rightarrow$\textsf{Set Up General Fields},
在\verb|General:|后加入\verb|language|就可以了。

有时候不想要上标,那么可以这样 \cite{shaheshang},这个非常重要。

\section{代码高亮}
有些时候我们需要在论文中引入一段代码,用来衬托正文的内容,或者体现关键思路的实现。
在模板中,统一使用\texttt{listings}宏包,并且设置了基本的内容格式,并建议用户只
使用三个接口,分别控制:编程语言,行号以及边框。简洁达意即可,下面分别举例说明。

首先是设定语言,来一个C的,使用的是默认设置:
\begin{lstlisting}[language=C]
void sort(int arr[], int beg, int end)
{
  if (end > beg + 1)
  {
    int piv = arr[beg], l = beg + 1, r = end;
    while (l < r)
    {
      if (arr[l] <= piv)
        l++;
      else
        swap(&arr[l], &arr[--r]);
    }
    swap(&arr[--l], &arr[beg]);
    sort(arr, beg, l);
    sort(arr, r, end);
  }
}
\end{lstlisting}

当我们需要高亮Java代码,不需要行号,不需要边框时,可以:
\begin{lstlisting}[language=Java,numbers=none,frame=none]
// A program to display the message
// "Hello World!" on standard output

public class HelloWorld {
 
   public static void main(String[] args) {
      System.out.println("Hello World!");
   }
      
}   // end of class HelloWorld
\end{lstlisting}

细心的用户可能发现,行号被放在了正文框之外,事实上这样是比较美观的,如果有些用户希望在正文框架之内布置所有内容,
可以:
\begin{lstlisting}[language=perl,xleftmargin=2em,framexleftmargin=1.5em]
#!/usr/bin/perl
print "Hello, world!\n";
\end{lstlisting}

好了,就这么多,\texttt{listings}宏包的功能很强大也很复杂,如果需要自己定制,可以
查看其手册,耐心阅读总会找到答案。\textbf{注意:} 当前中文注释的处理还不是很完善,
对于注释请妥善处理。在本模板中,推荐算法环境或者去掉中文的listings代码环境。如果
需要包含中文注释,不要求代码高亮~,就用\texttt{code}环境,这个环境是Verbatim的定制
版,调用的fancyvbr宏包,用户可在myscnu.sty中修改。

\begin{code}
public class HelloWorld {
   public static void main(String[] args) {
      System.out.println("Hello World!");
   }
}   // 世界,你好!
\end{code}

\section{中文习惯}
\label{sec:chinese}

对于itermize过大的行间距,用户可以使用compactitem环境来替代,但是模板中不进行默认替代,
因为只有用户真正发现列表不好看才会找到这里。

对于中文双引号,可以直接使用全角的\verb|“|和\verb|”|。但是英文则不行!英文的引
号用法请自行Google,或者阅读我的
\href{https://dl.dropbox.com/u/49734213/LaTeX%E6%9C%AD%E8%AE%B0.pdf}{\LaTeX{}}札
记。

中文破折号为一个两个字宽垂直居中的直线,输入法直接得到的破折号没有垂直居中(——),
这看起来不舒服。所以模板中定义了一个破折号的命令 \verb|\pozhehao|,请看:

艰苦奋斗、严谨治学、求实创新、为人师表\hfill \pozhehao{}华南师范大学校训